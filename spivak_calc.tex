\documentclass[11pt]{article}
\usepackage{amsthm}
\usepackage{amsmath}
\usepackage{calc}
\usepackage{thmtools}
\usepackage{mathtools}
\usepackage{bm}

\newcommand{\setnumref}[2]{\setcounter{#1}{#2}\addtocounter{#1}{-1}}
\renewcommand{\thepart}{\Roman{part}}
\renewcommand{\thesection}{\thepart~\arabic{section}}
\newcommand{\partref}[1]{\setnumref{part}{#1}\part}
\newcommand{\sectionref}[1]{\setnumref{section}{#1}\section}
\newcommand{\subsectionref}[1]{\setnumref{subsection}{#1}\subsection}
%$\newcommand{\subsectionref}[1]{\renewcommand{\thesubsection}{\thesection.#1}\subsection}

\declaretheoremstyle[headfont=\sc,spaceabove=3em]{axiomstyle}

\declaretheorem[style=axiomstyle, name=Axiom]{inneraxiomref}
\newenvironment{axiomref}[1]
{\renewcommand\theinneraxiomref{#1}\inneraxiomref}
{\endinneraxiomref}

\declaretheorem[style=axiomstyle, name=Theorem]{innerthmref}
\newenvironment{theoremref}[1]
{\renewcommand\theinnerthmref{#1}\innerthmref}
{\endinnerthmref}

\declaretheorem[name=Problem]{innerproblem}
\newenvironment{problem}[1]
{\renewcommand\theinnerproblem{#1}\innerproblem}
{\endinnerproblem}

\declaretheoremstyle[headfont=\sc, notefont=\sc, notebraces={}{}]{defstyle}
\declaretheorem[style=defstyle, numbered=no]{definition}

\declaretheorem{lemma}

\declaretheorem[style=remark]{remark}

% Common set definitions (reals, rationals, etc)
\newcommand{\Rplus}{\ensuremath{\Real^+}}       % reals (positive)
\newcommand{\Real}{\ensuremath{\mathbf{R}}}     % reals

\newcommand{\Integer}{\ensuremath{\mathbf{Z}}}  % integers
\newcommand{\Zplus}{\ensuremath{\mathbf{P}}}    % integers (positive)

\newcommand{\Rational}{\ensuremath{\mathbf{Q}}} % rationals

\DeclarePairedDelimiter{\abs}{\lvert}{\rvert}
% flip meaning of starred and non-starred commands so non-starred resizes
\makeatletter
\let\oldabs\abs
\def\abs{\@ifstar{\oldabs}{\oldabs*}}
\makeatother

\newcommand{\evalat}{\Bigg\rvert}

\begin{document}

\tableofcontents

\sectionref{1}{Basic Properties of Numbers}

For any numbers $a$, $b$, and $c$:

\begin{axiomref}{P1}[Associative law for addition]
  $a + (b + c) = (a + b) + c$.
\end{axiomref}

\begin{axiomref}{P2}[Existence of an additive identity]
  $a + 0 = 0 + a = a$.
\end{axiomref}

\begin{axiomref}{P3}[Existence of additive inverses]
  $a + (-a) = (-a) + a = 0$.
\end{axiomref}

\begin{axiomref}{P4}[Commutative law for addition]
  $a + b = b + a$.
\end{axiomref}

\begin{axiomref}{P5}[Associative law for multiplication]
  $a \cdot (b \cdot c) = (a \cdot b) \cdot c$.
\end{axiomref}

\begin{axiomref}{P6}[Existence of a multiplicative identity]
  $a \cdot 1 = 1 \cdot a = a; 1 \neq 0$.
\end{axiomref}

\begin{axiomref}{P7}[Existence of multiplicative inverses]
  $a \cdot a^{-1} = a^{-1} \cdot a = 1$, for $a \neq 0$.
\end{axiomref}

\begin{axiomref}{P8}[Commutative law for multiplication]
  $a \cdot b = b \cdot a$.
\end{axiomref}

\begin{axiomref}{P9}[Distributive law]
  $a \cdot (b + c) = a \cdot b + a \cdot c$.
\end{axiomref}

In the following, $P$ is the collection of all positive numbers. That
is $P = \{p \in \Real | p > 0\}$.

\begin{axiomref}{P10}[Trichotomy law]
  For every number $a$, one and only of the following holds:
  \begin{enumerate}
    \item $a = 0$,
    \item $a$ is in the collection $P$,
    \item $-a$ is in the collection $P$.
  \end{enumerate}
\end{axiomref}

\begin{axiomref}{P11}[Closure under addition]
  If $a$ and $b$ are in $P$, then $a + b$ is in $P$.
\end{axiomref}

\begin{axiomref}{P12}[Closure under multiplication]
  If $a$ and $b$ are in $P$, then $a \cdot b$ is in $P$.
\end{axiomref}

Some symbol definitions:

\begin{itemize}
\item $a > b$ if $a - b$ is in P;
\item $a < b$ if $b > a$;
\item $a \geq b$ if $a > b$ or $a = b$;
\item $a \leq b$ if $a < b$ or $a = b$.
\end{itemize}

\sectionref{5}{Limits}

\begin{definition}
  The function \textbf{$f$ approaches the limit $l$ near $a$} means:
  for every $\varepsilon > 0$ there is some $\delta > 0$ such that,
  for all $x$, if $0 < \abs{x - a} < \delta$, then
  $\abs{f(x) - l} < \varepsilon.$

  \begin{remark}
    Usually the goal here is to express $delta$ as a function of
    $\varepsilon$. This way, any $\varepsilon$ that is picked has an
    associated $\delta$.
  \end{remark}
\end{definition}

\sectionref{6}{Continuous Functions}

\begin{definition}
  The function $f$ is \textbf{continuous at $a$} if
  \[
    \lim_{x \to a}{f(x)} = f(a).
  \]
\end{definition}

\begin{theoremref}{6-1}
  If $f$ and $g$ are continuous at $a$, then

  \begin{enumerate}
  \item $f + g$ is continuous at $a$,
  \item $f \cdot g$ is continuous at $a$.
  \item $1/g$ is continuous at $a$, if $g(a) \neq 0$.
  \end{enumerate}
\end{theoremref}

\sectionref{7}{Three Hard Theorems (Continuity on Intervals)}

\begin{theoremref}{7-1}[Intermediate Value Theorem (with 7-4 and 7-5)]
  If $f$ is continuous on $[a,b]$ and $f(a) < 0 < f(b)$, then there
  is some $x$ in $[a,b]$ such that $f(x) = 0$.
  \begin{remark}
    A continuous function crosses the x-axis if it starts below it and
    ends above it.
  \end{remark}
\end{theoremref}

\begin{theoremref}{7-2}
  If $f$ is continuous on $[a,b]$ then $f$ is bounded above on
  $[a,b]$, that is, there is some number $N$ such that $f(x) \leq N$
  for all $x$ in $[a,b]$.
  \begin{remark}
    The graph of $f$ lies below some horizontal line. In other words,
    all values of $f$ on an interval will be less than or equal to
    some number $N$.
  \end{remark}
\end{theoremref}

\begin{theoremref}{7-3}
  If $f$ is continuous on $[a,b]$, then there is some number $y$ in
  $[a,b]$ such that $f(y) \geq f(x)$ for all $x$ in $[a,b]$.
  \begin{remark}
    There is a maximum value of a continuous function $f$ on a closed
    interval.
  \end{remark}
\end{theoremref}

\begin{theoremref}{7-4}
  If $f$ is continuous on $[a,b]$ and $f(a) < c < f(b)$, then there is
  some $x$ in $[a,b]$ such that $f(x) = c$.
\end{theoremref}

\begin{theoremref}{7-5}
  If $f$ is continuous on $[a,b]$ and $f(a) > c > f(b)$, then there is
  some $x$ in $[a,b]$ such that $f(x) = c$.
\end{theoremref}

\begin{remark}
  \emph{Intermediate value theorem}. If a continuous function on an
  interval takes on two values, it takes on every value in between. A
  simple consequence of Theorems 7-4 and 7-5 together.
\end{remark}

\begin{theoremref}{7-6}
  If $f$ is continuous on $[a,b]$, then $f$ is bounded below on
  $[a,b]$, that is, there is some number $N$ such that $f(x) \geq N$
  for all $x$ in $[a,b]$.
\end{theoremref}

\begin{remark}
  A continuous function is bounded from above and below, as a
  consequence of Theorems 7-2 and 7-6.
\end{remark}

\begin{theoremref}{7-7}
  If $f$ is continuous on $[a,b]$, thne there is some $y$ in $[a,b]$
  such that $f(y) \leq f(x)$ for all $x$ in $[a,b]$.
  \begin{remark}
    There is a minimum value of a continuous function on a closed interval.
  \end{remark}
\end{theoremref}

\begin{theoremref}{7-8}
  Every positive number has a square root. In other words, if $\alpha
  > 0$, then there is some number $x$ such that $x^2 = \alpha$.
\end{theoremref}

\begin{theoremref}{7-9}
  If $n$ is odd, then any equation

  \[
    x^n + a_{n-1}x^{n-1} + \cdots + a_0 = 0
  \]

has a root.
\end{theoremref}

\begin{theoremref}{7-10}
  If $n$ is even and $f(x) = x^n + a_{n-1}x^{n-1} + \cdots + a_0$,
  then there is a number $y$ such that $f(y) \leq f(x)$ for all $x$.
\end{theoremref}

\begin{theoremref}{7-11}
  Consider the equation

  \[
  x^n + a_{n-1}x^{n-1} + \cdots + a_0 = c,
  \]

  and suppose $n$ is even. Then there is a number $m$ such that the
  equation has a solution for $c \geq m$ and has no solution for $c < m$.
\end{theoremref}

\sectionref{9}{Derivatives}

\begin{definition}
  The function $f$ is \textbf{differentiable at $a$} if
  \[
    \lim_{h \to 0}{\frac{f(a+h) - f(a)}{h}} \text{ exists.}
  \]

  In this case the limit is denoted by $\bm{f'(a)}$ and is called the
  \textbf{derivative of $f$ at $a$}. (We also say that $f$ is
  \textbf{differentiable} if $f$ is differentiable at $a$ for every
  $a$ in the domain of $f$.)
\end{definition}

\begin{theoremref}{9-1}
  If $f$ is differentiable at $a$, then $f$ is continuous at $a$.
  \begin{remark}
    The converse is \emph{not true}. A differentiable function is
    continuous, but a continuous function may not be differentiable.
  \end{remark}
\end{theoremref}

\sectionref{10}{Differentiation}

\sectionref{11}{Significance of the Derivative}

\begin{definition}
  Let $f$ be a function and $A$ a set of numbers contained in the
  domain of $f$. A point $x$ in $A$ is a \textbf{maximum point} for
  $f$ on $A$ if

  \[
  f(x) \geq f(y) \quad \text{for every $y$ in $A$.}
  \]

  The number $f(x)$ itself is called a \textbf{maximum value} of $f$
  on $A$ (and we also say that $f$ ``has its maximum value on $A$ at
  $x$'').
\end{definition}

\begin{theoremref}{11-1}
  Left $f$ be any function defined on $(a,b)$. If $x$ is a maximum
  (or a minimum) point of $f$ on $(a,b)$, and $f$ is differentiable
  at $x$, then $f'(x) = 0$.
\end{theoremref}

\begin{theoremref}{11-4}[Mean Value Theorem]
  \emph{Very important!} If $f$ is continuous on $[a,b]$ and
  differentiable on $(a,b)$, then there is a number $x$ in $(a,b)$
  such that

  \[
  f'(x) = \frac{f(b) - f(a)}{b - a}.
  \]
\end{theoremref}

\begin{theoremref}{11-5}
  Suppose $f'(a) = 0$. If $f''(a) > 0$, then $f$ has a local maximum
  at $a$; if $f''(a) < 0$, then $f$ has a local maximum at $a$.
  \begin{remark}
    $f''(a) = 0$ is not informative; $f(a)$ may be a local maximum, a
    local minimum, or neither.
  \end{remark}
\end{theoremref}

\begin{remark}[Sketching a graph]
  Find:
  \begin{enumerate}
  \item critical points of $f$ (where $f'(x) = 0$)
  \item value of $f$ at critical points
  \item sign of $f'$ in the regions between critical points
  \item the numbers $x$ such that $f(x) = 0$
  \item behavior of $f(x)$ as $x$ becomes large or large negative
  \item check, if easy, if function is even or odd
  \end{enumerate}
\end{remark}

\begin{theoremref}{11-8}[Cauchy Mean Value Theorem]
  If $f$ and $g$ are continuous on $[a,b]$ and differentiable on $(a,
  b)$, then there is a number $x$ in $(a, b)$ such that

  \[
  [f(b) - f(a)]g'(x) = [g(b)-g(a)]f'(x)
  \]

  (If $g(b) \neq g(a)$ and $g'(x) \neq 0$, this equation can be
  written

  \[
  \frac{f(b) - f(a)}{g(b) - g(a)} = \frac{f'(x)}{g'(x)}. \quad )
  \]
\end{theoremref}

\begin{theoremref}{11-9}[L'H\^{o}pital's Rule]
  Suppose that

  \[
  \lim_{x \to a}{f(x)} = 0 \quad \text{and} \quad \lim_{x \to a}{g(x)}
  = 0,
  \]

  and suppose also that $\lim_{x \to a}{f'(x)/g'(x)}$ exists. Then
  $\lim_{x \to a}{f(x)/g(x)}$ exists, and

  \[
  \lim_{x \to a}{\frac{f(x)}{g(x)}} = \lim_{x \to a}{\frac{f'(x)}{g'(x)}}.
  \]
\end{theoremref}

\sectionref{14}{Fundamental Theorem of Calculus}

\begin{theoremref}{14-1}[The First Fundamental Theorem of Calculus]
  Let $f$ be integrable on $[a, b]$, and define $F$ on $[a,b]$ by
  \[
    F(x) = \int_{a}^{x}{f}.
  \]
  If $f$ is continuous at $c$ in $[a,b]$, then $F$ is differentiable
  at $c$, and
  \[
    F'(c) = f(c).
  \]
\end{theoremref}

\begin{theoremref}{19-2}[The Second Fundamental Theorem of Calculus]
  If $f$ is integrable on $[a,b]$ and $f = g'$ for some function $g$,
  then
  \[
    \int_{a}^{b}{f} = g(b) - g(a).
  \]
\end{theoremref}

\sectionref{19}{Integration in Elementary Terms (Indefinite
  Integration)}

Notation and some terms:

\[
F(x)\evalat_{a}^{b} = F(b) - F(a) \qquad \text{, by defintion}
\]

A function $F$ that satisfies $F' = f$ is a \textbf{primitive} of $f$.

\begin{theoremref}{19-1}[Integration by Parts]
  If $f'$ and $g'$ are continuous, then

  \begin{gather}
    \int{fg'} = fg - \int{f'g} \\
    \int{f(x)g'(x)\, dx} = f(x)g(x) - \int{f'(x)g(x)\, dx} \\
    \int_{a}^{b}{f(x)g'(x)\, dx = f(x)g(x)\evalat_a^b - \int_a^b{f'(x)g(x)\, dx}}
  \end{gather}

  \begin{remark}
    Consequence of the product rule for differentiation $(fg)' = f'g +
    fg'$, which can be written

    \begin{align*}
      fg' &= (fg)' - f'g \\
      \int{fg'} &= \int{(fg)'} - \int{f'g},
    \end{align*}

    and $\int{(fg)'} = fg$, so

    \begin{align*}
      \int{fg'} = fg - \int{f'g}
    \end{align*}

  \end{remark}
\end{theoremref}

\begin{theoremref}{19-2}[The Substitution Formula]
  If $f$ and $g'$ are continuous, then

  \begin{gather}
  \int_{g(a)}^{g(b)} f = \int_{a}^{b} (f \circ g) \cdot g' \\
  \int_{g(a)}^{g(b)} f(u)\, du = \int_{a}^{b} f(\underbrace{g(x)}_{u}) \cdot \underbrace{g'(x)\, dx}_{du}.
\end{gather}

\begin{remark}
  A consequence of the Chain Rule $(f \circ g)' = (f' \circ g)
  \cdot g'$
\end{remark}

\end{theoremref}

\subsection{Indefinite Integrals}

\begin{align}
  &\int{a \, dx} = ax \\
  &\int{x^{n}\, dx} = \frac{x^{n+1}}{n+1}, \quad n \neq -1 \\
  &\int{\frac{1}{x}\, dx} = \log{x} \\
  &\int{e^{x}\, dx} = e^{x} \\
  &\int{\sin x\, dx} = -\cos x \\
  &\int{\cos x\, dx} = \sin x \\
  &\int{\sec^2 x\, dx} = \tan x \\
  &\int{\sec x \tan x \, dx} = \sec x \\
  &\int{\frac{dx}{1 + x^{2}}} = \arctan x \\
  &\int{\frac{dx}{\sqrt{1 - x^{2}}}} = \arcsin x \\
\end{align}

\end{document}

%%% Local Variables:
%%% mode: latex
%%% TeX-master: t
%%% End:
